\documentclass[12pt]{article}
\usepackage{amsfonts, amssymb,amsmath}
%\usepackage{graphicx}


\begin{document}

\begin{titlepage}
   \begin{center}
       \vspace*{0.5cm}

       \LARGE{\textbf{Lecture Notes}}

       \vspace{1cm}
        \Large{Quantum Mechanics II}
            
       \vspace{1cm}
		\small{Gavin Kerr (B00801584)} \\
		\vfill
%		\includegraphics[scale=0.65]{dal_logo2.png}
       \vfill
           \large{ PHYC 0000}
            
       \vspace{0.8cm}
     
            
       Physics and Atmospheric Science\\
       Dalhousie University\\
            
   \end{center}
\end{titlepage}


\section{Definitions}
Linear Vector Space: Collection of objects
\begin{align*}
|v\rangle + |w\rangle \ \varepsilon& \, V
\\
a(|v\rangle + |w\rangle) =& a|v\rangle + a|w\rangle
\\
a(b|v\rangle) =& b(a|v\rangle)
\\
|v\rangle + |w\rangle =& |w\rangle + |v\rangle
\\
|v\rangle + (|w\rangle + |x\rangle) =& 
(v\rangle + |w\rangle) + |x\rangle
\\
\text{There needs to be a null vector} (|0\rangle) \rightarrow& |v\rangle + |0\rangle = 0 
\\
\text{For every vector (v) there is an inverse} &
|v\rangle + |-v\rangle = |0\rangle
\end{align*}


\subsection*{2022-09-07}

\section{Vector Spaces}

\textbf{Def.} Linear Vector Space: Collection of objects which follow the rules below
\begin{align*}
|v\rangle + |w\rangle \ \varepsilon& \, V
\\
a(|v\rangle + |w\rangle) =& a|v\rangle + a|w\rangle
\\
a(b|v\rangle) =& b(a|v\rangle)
\\
|v\rangle + |w\rangle =& |w\rangle + |v\rangle
\\
|v\rangle + (|w\rangle + |x\rangle) =& 
(v\rangle + |w\rangle) + |x\rangle
\\
\text{There needs to be a null vector} (|0\rangle) \rightarrow& |v\rangle + |0\rangle = 0 
\\
\text{For every vector (v) there is an inverse} &
|v\rangle + |-v\rangle = |0\rangle
\end{align*}





















\end{document}